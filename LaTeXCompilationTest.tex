\documentclass{article}

% Required packages for advanced math and graphics
\usepackage{amsmath}
\usepackage{amsfonts}
\usepackage{amssymb}
\usepackage{graphicx}
\usepackage[english]{babel}

% Document Information
\title{Comprehensive LaTeX Feature Test}
\author{MarkFlow Test Suite}
\date{\today}

\begin{document}

\maketitle

\section{Introduction}
This document serves as a test case to verify the LaTeX compilation capabilities of the MarkFlow editor. It includes a variety of common mathematical notations, environments, and text formatting options.

\section{Text Formatting}
Here are some basic text formatting examples:
\begin{itemize}
    \item This is standard text.
    \item \textbf{This text is bold.}
    \item \textit{This text is italic.}
    \item \texttt{This text is in a monospaced font.}
\end{itemize}

\section{Mathematical Notations}

\subsection{Inline Mathematics}
Inline math is used for expressions within a line of text, such as the famous mass-energy equivalence equation, $E = mc^2$. We can also include Greek letters like $\alpha, \beta, \gamma$ and symbols like $\hbar$.

\subsection{Display Mathematics (Unnumbered)}
Display math centers the equation on its own line.
\[
\int_{-\infty}^{\infty} e^{-x^2} \,dx = \sqrt{\pi}
\]

\subsection{Display Mathematics (Numbered)}
For equations that need to be referenced, we use the `equation` environment.
\begin{equation}
    \sin^2(\theta) + \cos^2(\theta) = 1
\end{equation}

\section{Advanced Mathematical Structures}

\subsection{Fractions and Roots}
Here is an example of a fraction and a square root:
\[
f(x) = \frac{x^2 - 1}{\sqrt{x+1}}
\]

\subsection{Matrices and Vectors}
Matrices can be created using the `pmatrix` environment from the `amsmath` package.
\[
\mathbf{A} = \begin{pmatrix}
a & b \\
c & d
\end{pmatrix}
\quad
\mathbf{v} = \begin{pmatrix}
x \\
y
\end{pmatrix}
\]

\subsection{Aligned Equations}
The `align` environment is perfect for aligning multiple equations, such as Maxwell's Equations.
\begin{align}
\nabla \cdot \mathbf{E} &= \frac{\rho}{\varepsilon_0} \label{eq:gauss_law} \\
\nabla \cdot \mathbf{B} &= 0 \\
\nabla \times \mathbf{E} &= -\frac{\partial \mathbf{B}}{\partial t} \\
\nabla \times \mathbf{B} &= \mu_0 \left( \mathbf{J} + \varepsilon_0 \frac{\partial \mathbf{E}}{\partial t} \right)
\end{align}
We can reference Gauss's Law using its label: Eq. \ref{eq:gauss_law}.

\subsection{The Schrödinger Equation}
A fundamental equation in quantum mechanics:
\begin{equation}
i\hbar\frac{\partial}{\partial t}\Psi(\mathbf{r},t) = \hat{H}\Psi(\mathbf{r},t)
\end{equation}

\section{Lists and Tables}
\subsection{Enumerated List}
\begin{enumerate}
    \item First item: The limit is $\lim_{x \to 0} \frac{\sin(x)}{x} = 1$.
    \item Second item: The sum is $\sum_{n=1}^{\infty} \frac{1}{n^2} = \frac{\pi^2}{6}$.
    \item Third item.
\end{enumerate}

\subsection{Table}
A simple table structure.
\begin{center}
\begin{tabular}{|l|c|r|}
\hline
\textbf{Name} & \textbf{Symbol} & \textbf{Value} \\
\hline
Pi & $\pi$ & 3.14159 \\
Euler's Number & $e$ & 2.71828 \\
Golden Ratio & $\phi$ & 1.61803 \\
\hline
\end{tabular}
\end{center}

\section{Conclusion}
If this document renders correctly, the LaTeX compiler is functioning as expected and supports a wide range of essential features.

\end{document}
